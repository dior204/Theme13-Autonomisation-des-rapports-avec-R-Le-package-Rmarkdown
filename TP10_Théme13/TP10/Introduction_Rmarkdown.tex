% Options for packages loaded elsewhere
\PassOptionsToPackage{unicode}{hyperref}
\PassOptionsToPackage{hyphens}{url}
\documentclass[
]{article}
\usepackage{xcolor}
\usepackage[margin=1in]{geometry}
\usepackage{amsmath,amssymb}
\setcounter{secnumdepth}{-\maxdimen} % remove section numbering
\usepackage{iftex}
\ifPDFTeX
  \usepackage[T1]{fontenc}
  \usepackage[utf8]{inputenc}
  \usepackage{textcomp} % provide euro and other symbols
\else % if luatex or xetex
  \usepackage{unicode-math} % this also loads fontspec
  \defaultfontfeatures{Scale=MatchLowercase}
  \defaultfontfeatures[\rmfamily]{Ligatures=TeX,Scale=1}
\fi
\usepackage{lmodern}
\ifPDFTeX\else
  % xetex/luatex font selection
\fi
% Use upquote if available, for straight quotes in verbatim environments
\IfFileExists{upquote.sty}{\usepackage{upquote}}{}
\IfFileExists{microtype.sty}{% use microtype if available
  \usepackage[]{microtype}
  \UseMicrotypeSet[protrusion]{basicmath} % disable protrusion for tt fonts
}{}
\makeatletter
\@ifundefined{KOMAClassName}{% if non-KOMA class
  \IfFileExists{parskip.sty}{%
    \usepackage{parskip}
  }{% else
    \setlength{\parindent}{0pt}
    \setlength{\parskip}{6pt plus 2pt minus 1pt}}
}{% if KOMA class
  \KOMAoptions{parskip=half}}
\makeatother
\usepackage{color}
\usepackage{fancyvrb}
\newcommand{\VerbBar}{|}
\newcommand{\VERB}{\Verb[commandchars=\\\{\}]}
\DefineVerbatimEnvironment{Highlighting}{Verbatim}{commandchars=\\\{\}}
% Add ',fontsize=\small' for more characters per line
\usepackage{framed}
\definecolor{shadecolor}{RGB}{248,248,248}
\newenvironment{Shaded}{\begin{snugshade}}{\end{snugshade}}
\newcommand{\AlertTok}[1]{\textcolor[rgb]{0.94,0.16,0.16}{#1}}
\newcommand{\AnnotationTok}[1]{\textcolor[rgb]{0.56,0.35,0.01}{\textbf{\textit{#1}}}}
\newcommand{\AttributeTok}[1]{\textcolor[rgb]{0.13,0.29,0.53}{#1}}
\newcommand{\BaseNTok}[1]{\textcolor[rgb]{0.00,0.00,0.81}{#1}}
\newcommand{\BuiltInTok}[1]{#1}
\newcommand{\CharTok}[1]{\textcolor[rgb]{0.31,0.60,0.02}{#1}}
\newcommand{\CommentTok}[1]{\textcolor[rgb]{0.56,0.35,0.01}{\textit{#1}}}
\newcommand{\CommentVarTok}[1]{\textcolor[rgb]{0.56,0.35,0.01}{\textbf{\textit{#1}}}}
\newcommand{\ConstantTok}[1]{\textcolor[rgb]{0.56,0.35,0.01}{#1}}
\newcommand{\ControlFlowTok}[1]{\textcolor[rgb]{0.13,0.29,0.53}{\textbf{#1}}}
\newcommand{\DataTypeTok}[1]{\textcolor[rgb]{0.13,0.29,0.53}{#1}}
\newcommand{\DecValTok}[1]{\textcolor[rgb]{0.00,0.00,0.81}{#1}}
\newcommand{\DocumentationTok}[1]{\textcolor[rgb]{0.56,0.35,0.01}{\textbf{\textit{#1}}}}
\newcommand{\ErrorTok}[1]{\textcolor[rgb]{0.64,0.00,0.00}{\textbf{#1}}}
\newcommand{\ExtensionTok}[1]{#1}
\newcommand{\FloatTok}[1]{\textcolor[rgb]{0.00,0.00,0.81}{#1}}
\newcommand{\FunctionTok}[1]{\textcolor[rgb]{0.13,0.29,0.53}{\textbf{#1}}}
\newcommand{\ImportTok}[1]{#1}
\newcommand{\InformationTok}[1]{\textcolor[rgb]{0.56,0.35,0.01}{\textbf{\textit{#1}}}}
\newcommand{\KeywordTok}[1]{\textcolor[rgb]{0.13,0.29,0.53}{\textbf{#1}}}
\newcommand{\NormalTok}[1]{#1}
\newcommand{\OperatorTok}[1]{\textcolor[rgb]{0.81,0.36,0.00}{\textbf{#1}}}
\newcommand{\OtherTok}[1]{\textcolor[rgb]{0.56,0.35,0.01}{#1}}
\newcommand{\PreprocessorTok}[1]{\textcolor[rgb]{0.56,0.35,0.01}{\textit{#1}}}
\newcommand{\RegionMarkerTok}[1]{#1}
\newcommand{\SpecialCharTok}[1]{\textcolor[rgb]{0.81,0.36,0.00}{\textbf{#1}}}
\newcommand{\SpecialStringTok}[1]{\textcolor[rgb]{0.31,0.60,0.02}{#1}}
\newcommand{\StringTok}[1]{\textcolor[rgb]{0.31,0.60,0.02}{#1}}
\newcommand{\VariableTok}[1]{\textcolor[rgb]{0.00,0.00,0.00}{#1}}
\newcommand{\VerbatimStringTok}[1]{\textcolor[rgb]{0.31,0.60,0.02}{#1}}
\newcommand{\WarningTok}[1]{\textcolor[rgb]{0.56,0.35,0.01}{\textbf{\textit{#1}}}}
\usepackage{longtable,booktabs,array}
\usepackage{calc} % for calculating minipage widths
% Correct order of tables after \paragraph or \subparagraph
\usepackage{etoolbox}
\makeatletter
\patchcmd\longtable{\par}{\if@noskipsec\mbox{}\fi\par}{}{}
\makeatother
% Allow footnotes in longtable head/foot
\IfFileExists{footnotehyper.sty}{\usepackage{footnotehyper}}{\usepackage{footnote}}
\makesavenoteenv{longtable}
\usepackage{graphicx}
\makeatletter
\newsavebox\pandoc@box
\newcommand*\pandocbounded[1]{% scales image to fit in text height/width
  \sbox\pandoc@box{#1}%
  \Gscale@div\@tempa{\textheight}{\dimexpr\ht\pandoc@box+\dp\pandoc@box\relax}%
  \Gscale@div\@tempb{\linewidth}{\wd\pandoc@box}%
  \ifdim\@tempb\p@<\@tempa\p@\let\@tempa\@tempb\fi% select the smaller of both
  \ifdim\@tempa\p@<\p@\scalebox{\@tempa}{\usebox\pandoc@box}%
  \else\usebox{\pandoc@box}%
  \fi%
}
% Set default figure placement to htbp
\def\fps@figure{htbp}
\makeatother
\setlength{\emergencystretch}{3em} % prevent overfull lines
\providecommand{\tightlist}{%
  \setlength{\itemsep}{0pt}\setlength{\parskip}{0pt}}
\usepackage{hyperref}
\usepackage{amsmath}
\usepackage{amssymb}
\usepackage{graphicx}
\usepackage{fontspec}
\usepackage{xcolor}
\usepackage{tikz}
\definecolor{blue}{RGB}{0, 0, 100}
\setmainfont{Times New Roman}
\setsansfont{Times New Roman}
\setmonofont{Courier New}
\usepackage[margin=1in]{geometry}
\usepackage{titlesec}
\titleformat{\section}{\Huge\bfseries\color{blue}}{\thesection}{1em}{}
\titleformat{\subsection}{\huge\bfseries\color{blue}}{\thesubsection}{1em}{}
\titleformat{\subsubsection}{\LARGE\bfseries\color{blue}}{\thesubsubsection}{1em}{}
\usepackage{tocloft}
\renewcommand{\cftsecfont}{\small}
\renewcommand{\cftsubsecfont}{\footnotesize}
\renewcommand{\cftsecpagefont}{\small}
\renewcommand{\cftsubsecpagefont}{\footnotesize}
\renewcommand{\cftsecleader}{\cftdotfill{\cftdotsep}}
\usepackage{fontspec}
\usepackage{multirow}
\usepackage{multicol}
\usepackage{colortbl}
\usepackage{hhline}
\newlength\Oldarrayrulewidth
\newlength\Oldtabcolsep
\usepackage{longtable}
\usepackage{array}
\usepackage{hyperref}
\usepackage{float}
\usepackage{wrapfig}
\usepackage{booktabs}
\usepackage{pdflscape}
\usepackage{tabu}
\usepackage{threeparttable}
\usepackage{threeparttablex}
\usepackage[normalem]{ulem}
\usepackage{makecell}
\usepackage{xcolor}
\usepackage{bookmark}
\IfFileExists{xurl.sty}{\usepackage{xurl}}{} % add URL line breaks if available
\urlstyle{same}
\hypersetup{
  hidelinks,
  pdfcreator={LaTeX via pandoc}}

\author{}
\date{\vspace{-2.5em}}

\begin{document}

\begin{titlepage}
    \begin{center}
    % Début de la bordure avec TikZ
    \begin{tikzpicture}[remember picture, overlay] % overlay = force TikZ à dessiner par-dessus le contenu existant de la page, plutôt que de réserver un espace supplémentaire pour le dessin, remember = lorsque vous voulez dessiner par rapport à la position exacte de la page
    % Définir une couleur élégante
    \definecolor{blue}{RGB}{0, 100, 0}
    
    % Dessiner une bordure élégante avec coins arrondis
    \draw[
        line width=5pt, % Épaisseur du trait
        black, % Couleur de la bordure
        rounded corners=15pt, % Coins arrondis
        double, % Bordure double
        double distance=2pt % Espacement entre les deux lignes
    ] 
        ([xshift=10pt, yshift=-10pt]current page.north west) rectangle
        ([xshift=-10pt, yshift=10pt]current page.south east);
    \end{tikzpicture}
        \includegraphics[width=7cm]{Figures/LOGO1.jpeg} \\[0.1cm]
        \includegraphics[width=6cm]{Figures/LOGO2.jpeg} \\[0.1cm]
        
        \textbf{\large Agence nationale de la Statistique et de la Démographie (ANSD)}\\[0.2cm]
        
        \includegraphics[width=4cm]{Figures/LOGO3.jpeg} \\[0.1cm]
        
        \textbf{\large Ecole nationale de la Statistique et de l'Analyse économique Pierre Ndiaye (ENSAE)}\\[0.4cm]
        
        \textit{\LARGE Semestre 2 : Projet statistique sous R }\\[0.3cm]
        \textbf{\Huge \color{blue} \textsf{Partie 1 : Introduction à Rmarkdown}}\\[0.2cm]
        
        \begin{minipage}{0.5\textwidth}
    \begin{flushleft} \large
        \emph{\textsf{Rédigé par :}}\\
        \textbf{Khadidiatou Diakhaté}\\
        \textbf{Haba Fromo Francis}\\
        \textbf{Dior Mbengue}\\
    \end{flushleft}
\end{minipage}
        \hfill
        \begin{minipage}{0.4\textwidth}
            \begin{flushright} \large
                \emph{\textsf{Sous la supervision de :}} \\
                \textbf{M. Aboubacre HEMA}\\
                \textit{Research Analyst }
            \end{flushright}
        \end{minipage}

        \vfill 

        {\large \textsf{Année scolaire : 2024/2025}}\\[0.5cm]
        
    \end{center}
\end{titlepage}

\section{Créer un fichier
Rmarkdown}\label{cruxe9er-un-fichier-rmarkdown}

\includegraphics[width=5.32292in,height=\textheight,keepaspectratio]{images/clipboard-372006340.png}

\section{Entête de Rmarkdown YAML(Yet Another Markup
Language)}\label{entuxeate-de-rmarkdown-yamlyet-another-markup-language}

\includegraphics[width=4.29167in,height=\textheight,keepaspectratio]{images/Capture d'écran 2025-04-16 223242.png}

Nous avons le titre du document R, l'auteur du document, la date où le
fichier a été créé et le type du document de sortie( Word, PPT, PDF,
Prettydoc, etc).

Pour générer les output, il suffit d'appuyer sur \emph{knit on save} ou
sur \emph{knit} qui propose le format HTML, PDF, Word et d'autres
options.

\includegraphics[width=4.45833in,height=\textheight,keepaspectratio]{images/clipboard-3418071425.png}

\textbf{OUTPUT Word}

\includegraphics[width=4.59375in,height=\textheight,keepaspectratio]{images/clipboard-358253334.png}

\textbf{OUTPUT HTML}

\includegraphics[width=4.59375in,height=\textheight,keepaspectratio]{images/clipboard-1337679791.png}

\textbf{OUTPUT PDF}

\includegraphics[width=4.84375in,height=\textheight,keepaspectratio]{images/clipboard-4183967572.png}

Pour avoir la sortie PDF, il faut installer LateX ou MicteX

Nous avons aussi l'output \textbf{Prettydoc}

\section{Les chunk options et les global
options}\label{les-chunk-options-et-les-global-options}

\subsection{chunk de configuration}\label{chunk-de-configuration}

Il est préférable de configurer le comportement par défaut Knitr pour
faire certains réglages comme la position et la taille des figures ou
l'affichage des lignes de codes(instructions sur R) ou non dans le
document de sortie. Ce chunk de configuration utilise cette fonction
pour définir les options par défaut applicables à tous les chunk en
dessous.

\begin{Shaded}
\begin{Highlighting}[]
\NormalTok{knitr}\SpecialCharTok{::}\NormalTok{opts\_chunk}\SpecialCharTok{$}\FunctionTok{set}\NormalTok{()}
\end{Highlighting}
\end{Shaded}

Aprés ce chunk, il est recommandé de faire un autre chunk pour charger
les packages et un autre pour charger les données qui seront utilisés
tout au long.

\begin{Shaded}
\begin{Highlighting}[]
\CommentTok{\#vérifier et installer les packages nécessaires}
\NormalTok{packages }\OtherTok{\textless{}{-}} \FunctionTok{c}\NormalTok{(}\StringTok{"haven"}\NormalTok{, }\StringTok{"utils"}\NormalTok{, }\StringTok{"dplyr"}\NormalTok{, }\StringTok{"gtsummary"}\NormalTok{, }\StringTok{"survey"}\NormalTok{,}\StringTok{"labelled"}\NormalTok{,}\StringTok{"compareDF"}\NormalTok{)}
\ControlFlowTok{for}\NormalTok{ (pkg }\ControlFlowTok{in}\NormalTok{ packages) \{}
  \ControlFlowTok{if}\NormalTok{ (}\SpecialCharTok{!}\FunctionTok{require}\NormalTok{(pkg, }\AttributeTok{character.only =} \ConstantTok{TRUE}\NormalTok{)) }\FunctionTok{install.packages}\NormalTok{(pkg, }\AttributeTok{dependencies =} \ConstantTok{TRUE}\NormalTok{)}
  \FunctionTok{library}\NormalTok{(pkg, }\AttributeTok{character.only =} \ConstantTok{TRUE}\NormalTok{)}
\NormalTok{\}}
\end{Highlighting}
\end{Shaded}

Dans la mesure du possible, il est préférable d'utiliser des chemins
relatifs pour les fichiers (exemple : ./donnees/mesdonnees.csv), plutôt
que des chemins absolus (exemple :
D:/chemin/vers/les/donnees/mesdonnees.csv).

Il peut arriver qu'il soit difficile d'accéder à des fichiers en
utilisant uniquement des chemins relatifs. Une solution de repli
consiste à définir le chemin absolu du dossier à un seul endroit dans le
code source du rapport, puis à l'utiliser de façon relative dans les
fonctions d'importation. Voici un exemple :

\begin{Shaded}
\begin{Highlighting}[]
\CommentTok{\# Définir UNE SEULE FOIS le répertoire des données}
\NormalTok{dossier\_donnees }\OtherTok{\textless{}{-}} \StringTok{"D:/chemin/vers/les/donnees/"}
\CommentTok{\# Charger les données avec un chemin composé}
\NormalTok{donnees1 }\OtherTok{\textless{}{-}} \FunctionTok{fread}\NormalTok{(}\FunctionTok{paste0}\NormalTok{(dossier\_donnees, }\StringTok{"mesdonnees1.csv"}\NormalTok{))}
\NormalTok{donnees2 }\OtherTok{\textless{}{-}} \FunctionTok{fread}\NormalTok{(}\FunctionTok{paste0}\NormalTok{(dossier\_donnees, }\StringTok{"mesdonnees2.csv"}\NormalTok{))}
\end{Highlighting}
\end{Shaded}

On peut aussi nommer un chunk

\begin{Shaded}
\begin{Highlighting}[]
\NormalTok{resultat }\OtherTok{\textless{}{-}} \DecValTok{1} \SpecialCharTok{+} \DecValTok{1}
\FunctionTok{print}\NormalTok{(resultat)}
\end{Highlighting}
\end{Shaded}

\begin{verbatim}
## [1] 4
\end{verbatim}

\pandocbounded{\includegraphics[keepaspectratio]{images/clipboard-160696737.png}}

\begin{Shaded}
\begin{Highlighting}[]
\DecValTok{2}\SpecialCharTok{+}\DecValTok{2}
\CommentTok{\# eval=FALSE veut dire qu\textquotesingle{}on montre les lignes de code mais pas l\textquotesingle{}output}
\end{Highlighting}
\end{Shaded}

\pandocbounded{\includegraphics[keepaspectratio]{images/clipboard-531356218.png}}

\begin{Shaded}
\begin{Highlighting}[]
\FunctionTok{library}\NormalTok{(tidyverse)}
\end{Highlighting}
\end{Shaded}

\begin{verbatim}
## -- Attaching core tidyverse packages ------------------------ tidyverse 2.0.0 --
## v dplyr     1.1.4     v readr     2.1.5
## v forcats   1.0.0     v stringr   1.5.1
## v ggplot2   3.5.1     v tibble    3.2.1
## v lubridate 1.9.4     v tidyr     1.3.1
## v purrr     1.0.2     
## -- Conflicts ------------------------------------------ tidyverse_conflicts() --
## x dplyr::filter() masks stats::filter()
## x dplyr::lag()    masks stats::lag()
## i Use the conflicted package (<http://conflicted.r-lib.org/>) to force all conflicts to become errors
\end{verbatim}

\begin{Shaded}
\begin{Highlighting}[]
\CommentTok{\# Pour que les messages générés sur le chunk ne soient pas affichés}
\end{Highlighting}
\end{Shaded}

\begin{Shaded}
\begin{Highlighting}[]
\FunctionTok{library}\NormalTok{(tidyverse)}
\CommentTok{\# Pour que les messages générés sur le chunk ne soient pas affichés}
\end{Highlighting}
\end{Shaded}

\includegraphics[width=4.91667in,height=\textheight,keepaspectratio]{images/clipboard-3495288595.png}

\pandocbounded{\includegraphics[keepaspectratio]{images/clipboard-3843030201.png}}

Dans cet exemple, il est demandé à R Markdown de ne pas inclure les
instructions R dans le document de sortie (echo = FALSE). Les warnings
et les messages d'informations n'apparaîtront pas non plus dans le
fichier de sortie. En revanche, les erreurs apparaîtront (error = TRUE).
Les figures sont centrées (fig.align = ``center'') et ont une largeur de
75\% de la largeur du texte (out.width = ``75\%'').

Nous avons encore

\pandocbounded{\includegraphics[keepaspectratio]{images/clipboard-2140684100.png}}

\section{Insertion d'éléments et de
bullet}\label{insertion-duxe9luxe9ments-et-de-bullet}

\begin{itemize}
\item
  Pour insérer une image sur Rmarkdown, on peut mettre un code du type
  \pandocbounded{\includegraphics[keepaspectratio]{images/clipboard-4183967572.png}}''
  ou juste copier l'image dans la partie \textbf{Visual}
\item
  Pour insérer un lien:{[}(\url{https://thegraph.org}){]}
\end{itemize}

\section{Inline Code}\label{inline-code}

Les inlines codes dans Rmarkdown permettent d'insérer directement des
résultats à l'intérieur du texte. Il sont utiles pour rendre le rapport
dynamique.

\begin{Shaded}
\begin{Highlighting}[]
\FunctionTok{max}\NormalTok{(women}\SpecialCharTok{$}\NormalTok{height)}
\end{Highlighting}
\end{Shaded}

\begin{verbatim}
## [1] 72
\end{verbatim}

The maximum women's height is 72.

\pandocbounded{\includegraphics[keepaspectratio]{images/clipboard-3684436073.png}}

\section{Les tableaux}\label{les-tableaux}

\begin{Shaded}
\begin{Highlighting}[]
\NormalTok{women }\CommentTok{\# sans recommandation de tableaux}
\end{Highlighting}
\end{Shaded}

\begin{verbatim}
##    height weight
## 1      58    115
## 2      59    117
## 3      60    120
## 4      61    123
## 5      62    126
## 6      63    129
## 7      64    132
## 8      65    135
## 9      66    139
## 10     67    142
## 11     68    146
## 12     69    150
## 13     70    154
## 14     71    159
## 15     72    164
\end{verbatim}

\pandocbounded{\includegraphics[keepaspectratio]{images/clipboard-261503643.png}}

\begin{Shaded}
\begin{Highlighting}[]
\CommentTok{\# Afficher des tableaux à partir des données de R}
\NormalTok{pacman}\SpecialCharTok{::}\FunctionTok{p\_load}\NormalTok{(flextable, gt, reactable)}
\NormalTok{flextable}\SpecialCharTok{::}\FunctionTok{flextable}\NormalTok{(women)}
\end{Highlighting}
\end{Shaded}

\global\setlength{\Oldarrayrulewidth}{\arrayrulewidth}

\global\setlength{\Oldtabcolsep}{\tabcolsep}

\setlength{\tabcolsep}{2pt}

\renewcommand*{\arraystretch}{1.5}



\providecommand{\ascline}[3]{\noalign{\global\arrayrulewidth #1}\arrayrulecolor[HTML]{#2}\cline{#3}}

\begin{longtable}[c]{|p{0.75in}|p{0.75in}}



\ascline{1.5pt}{666666}{1-2}

\multicolumn{1}{>{\raggedleft}m{\dimexpr 0.75in+0\tabcolsep}}{\textcolor[HTML]{000000}{\fontsize{11}{11}\selectfont{\global\setmainfont{Arial}{height}}}} & \multicolumn{1}{>{\raggedleft}m{\dimexpr 0.75in+0\tabcolsep}}{\textcolor[HTML]{000000}{\fontsize{11}{11}\selectfont{\global\setmainfont{Arial}{weight}}}} \\

\ascline{1.5pt}{666666}{1-2}\endfirsthead 

\ascline{1.5pt}{666666}{1-2}

\multicolumn{1}{>{\raggedleft}m{\dimexpr 0.75in+0\tabcolsep}}{\textcolor[HTML]{000000}{\fontsize{11}{11}\selectfont{\global\setmainfont{Arial}{height}}}} & \multicolumn{1}{>{\raggedleft}m{\dimexpr 0.75in+0\tabcolsep}}{\textcolor[HTML]{000000}{\fontsize{11}{11}\selectfont{\global\setmainfont{Arial}{weight}}}} \\

\ascline{1.5pt}{666666}{1-2}\endhead



\multicolumn{1}{>{\raggedleft}m{\dimexpr 0.75in+0\tabcolsep}}{\textcolor[HTML]{000000}{\fontsize{11}{11}\selectfont{\global\setmainfont{Arial}{58}}}} & \multicolumn{1}{>{\raggedleft}m{\dimexpr 0.75in+0\tabcolsep}}{\textcolor[HTML]{000000}{\fontsize{11}{11}\selectfont{\global\setmainfont{Arial}{115}}}} \\





\multicolumn{1}{>{\raggedleft}m{\dimexpr 0.75in+0\tabcolsep}}{\textcolor[HTML]{000000}{\fontsize{11}{11}\selectfont{\global\setmainfont{Arial}{59}}}} & \multicolumn{1}{>{\raggedleft}m{\dimexpr 0.75in+0\tabcolsep}}{\textcolor[HTML]{000000}{\fontsize{11}{11}\selectfont{\global\setmainfont{Arial}{117}}}} \\





\multicolumn{1}{>{\raggedleft}m{\dimexpr 0.75in+0\tabcolsep}}{\textcolor[HTML]{000000}{\fontsize{11}{11}\selectfont{\global\setmainfont{Arial}{60}}}} & \multicolumn{1}{>{\raggedleft}m{\dimexpr 0.75in+0\tabcolsep}}{\textcolor[HTML]{000000}{\fontsize{11}{11}\selectfont{\global\setmainfont{Arial}{120}}}} \\





\multicolumn{1}{>{\raggedleft}m{\dimexpr 0.75in+0\tabcolsep}}{\textcolor[HTML]{000000}{\fontsize{11}{11}\selectfont{\global\setmainfont{Arial}{61}}}} & \multicolumn{1}{>{\raggedleft}m{\dimexpr 0.75in+0\tabcolsep}}{\textcolor[HTML]{000000}{\fontsize{11}{11}\selectfont{\global\setmainfont{Arial}{123}}}} \\





\multicolumn{1}{>{\raggedleft}m{\dimexpr 0.75in+0\tabcolsep}}{\textcolor[HTML]{000000}{\fontsize{11}{11}\selectfont{\global\setmainfont{Arial}{62}}}} & \multicolumn{1}{>{\raggedleft}m{\dimexpr 0.75in+0\tabcolsep}}{\textcolor[HTML]{000000}{\fontsize{11}{11}\selectfont{\global\setmainfont{Arial}{126}}}} \\





\multicolumn{1}{>{\raggedleft}m{\dimexpr 0.75in+0\tabcolsep}}{\textcolor[HTML]{000000}{\fontsize{11}{11}\selectfont{\global\setmainfont{Arial}{63}}}} & \multicolumn{1}{>{\raggedleft}m{\dimexpr 0.75in+0\tabcolsep}}{\textcolor[HTML]{000000}{\fontsize{11}{11}\selectfont{\global\setmainfont{Arial}{129}}}} \\





\multicolumn{1}{>{\raggedleft}m{\dimexpr 0.75in+0\tabcolsep}}{\textcolor[HTML]{000000}{\fontsize{11}{11}\selectfont{\global\setmainfont{Arial}{64}}}} & \multicolumn{1}{>{\raggedleft}m{\dimexpr 0.75in+0\tabcolsep}}{\textcolor[HTML]{000000}{\fontsize{11}{11}\selectfont{\global\setmainfont{Arial}{132}}}} \\





\multicolumn{1}{>{\raggedleft}m{\dimexpr 0.75in+0\tabcolsep}}{\textcolor[HTML]{000000}{\fontsize{11}{11}\selectfont{\global\setmainfont{Arial}{65}}}} & \multicolumn{1}{>{\raggedleft}m{\dimexpr 0.75in+0\tabcolsep}}{\textcolor[HTML]{000000}{\fontsize{11}{11}\selectfont{\global\setmainfont{Arial}{135}}}} \\





\multicolumn{1}{>{\raggedleft}m{\dimexpr 0.75in+0\tabcolsep}}{\textcolor[HTML]{000000}{\fontsize{11}{11}\selectfont{\global\setmainfont{Arial}{66}}}} & \multicolumn{1}{>{\raggedleft}m{\dimexpr 0.75in+0\tabcolsep}}{\textcolor[HTML]{000000}{\fontsize{11}{11}\selectfont{\global\setmainfont{Arial}{139}}}} \\





\multicolumn{1}{>{\raggedleft}m{\dimexpr 0.75in+0\tabcolsep}}{\textcolor[HTML]{000000}{\fontsize{11}{11}\selectfont{\global\setmainfont{Arial}{67}}}} & \multicolumn{1}{>{\raggedleft}m{\dimexpr 0.75in+0\tabcolsep}}{\textcolor[HTML]{000000}{\fontsize{11}{11}\selectfont{\global\setmainfont{Arial}{142}}}} \\





\multicolumn{1}{>{\raggedleft}m{\dimexpr 0.75in+0\tabcolsep}}{\textcolor[HTML]{000000}{\fontsize{11}{11}\selectfont{\global\setmainfont{Arial}{68}}}} & \multicolumn{1}{>{\raggedleft}m{\dimexpr 0.75in+0\tabcolsep}}{\textcolor[HTML]{000000}{\fontsize{11}{11}\selectfont{\global\setmainfont{Arial}{146}}}} \\





\multicolumn{1}{>{\raggedleft}m{\dimexpr 0.75in+0\tabcolsep}}{\textcolor[HTML]{000000}{\fontsize{11}{11}\selectfont{\global\setmainfont{Arial}{69}}}} & \multicolumn{1}{>{\raggedleft}m{\dimexpr 0.75in+0\tabcolsep}}{\textcolor[HTML]{000000}{\fontsize{11}{11}\selectfont{\global\setmainfont{Arial}{150}}}} \\





\multicolumn{1}{>{\raggedleft}m{\dimexpr 0.75in+0\tabcolsep}}{\textcolor[HTML]{000000}{\fontsize{11}{11}\selectfont{\global\setmainfont{Arial}{70}}}} & \multicolumn{1}{>{\raggedleft}m{\dimexpr 0.75in+0\tabcolsep}}{\textcolor[HTML]{000000}{\fontsize{11}{11}\selectfont{\global\setmainfont{Arial}{154}}}} \\





\multicolumn{1}{>{\raggedleft}m{\dimexpr 0.75in+0\tabcolsep}}{\textcolor[HTML]{000000}{\fontsize{11}{11}\selectfont{\global\setmainfont{Arial}{71}}}} & \multicolumn{1}{>{\raggedleft}m{\dimexpr 0.75in+0\tabcolsep}}{\textcolor[HTML]{000000}{\fontsize{11}{11}\selectfont{\global\setmainfont{Arial}{159}}}} \\





\multicolumn{1}{>{\raggedleft}m{\dimexpr 0.75in+0\tabcolsep}}{\textcolor[HTML]{000000}{\fontsize{11}{11}\selectfont{\global\setmainfont{Arial}{72}}}} & \multicolumn{1}{>{\raggedleft}m{\dimexpr 0.75in+0\tabcolsep}}{\textcolor[HTML]{000000}{\fontsize{11}{11}\selectfont{\global\setmainfont{Arial}{164}}}} \\

\ascline{1.5pt}{666666}{1-2}



\end{longtable}



\arrayrulecolor[HTML]{000000}

\global\setlength{\arrayrulewidth}{\Oldarrayrulewidth}

\global\setlength{\tabcolsep}{\Oldtabcolsep}

\renewcommand*{\arraystretch}{1}

\begin{Shaded}
\begin{Highlighting}[]
\CommentTok{\# gt::gt(women)}
\CommentTok{\# reactable::reactable(women)}
\end{Highlighting}
\end{Shaded}

Un autre exemple

\begin{Shaded}
\begin{Highlighting}[]
\FunctionTok{head}\NormalTok{(iris)}
\end{Highlighting}
\end{Shaded}

\begin{verbatim}
##   Sepal.Length Sepal.Width Petal.Length Petal.Width Species
## 1          5.1         3.5          1.4         0.2  setosa
## 2          4.9         3.0          1.4         0.2  setosa
## 3          4.7         3.2          1.3         0.2  setosa
## 4          4.6         3.1          1.5         0.2  setosa
## 5          5.0         3.6          1.4         0.2  setosa
## 6          5.4         3.9          1.7         0.4  setosa
\end{verbatim}

\begin{Shaded}
\begin{Highlighting}[]
\CommentTok{\# cela affichera les premiéres lignes de iris}
\end{Highlighting}
\end{Shaded}

\begin{Shaded}
\begin{Highlighting}[]
\NormalTok{knitr}\SpecialCharTok{::}\FunctionTok{kable}\NormalTok{(}\FunctionTok{head}\NormalTok{(iris), }\AttributeTok{caption =} \StringTok{"Premières lignes du jeu de\_données\_iris"}\NormalTok{)}
\end{Highlighting}
\end{Shaded}

\begin{longtable}[]{@{}rrrrl@{}}
\caption{Premières lignes du jeu de\_données\_iris}\tabularnewline
\toprule\noalign{}
Sepal.Length & Sepal.Width & Petal.Length & Petal.Width & Species \\
\midrule\noalign{}
\endfirsthead
\toprule\noalign{}
Sepal.Length & Sepal.Width & Petal.Length & Petal.Width & Species \\
\midrule\noalign{}
\endhead
\bottomrule\noalign{}
\endlastfoot
5.1 & 3.5 & 1.4 & 0.2 & setosa \\
4.9 & 3.0 & 1.4 & 0.2 & setosa \\
4.7 & 3.2 & 1.3 & 0.2 & setosa \\
4.6 & 3.1 & 1.5 & 0.2 & setosa \\
5.0 & 3.6 & 1.4 & 0.2 & setosa \\
5.4 & 3.9 & 1.7 & 0.4 & setosa \\
\end{longtable}

\begin{Shaded}
\begin{Highlighting}[]
\CommentTok{\# On peut améliorer les tableaux avec cette variable}
\end{Highlighting}
\end{Shaded}

\begin{Shaded}
\begin{Highlighting}[]
\FunctionTok{library}\NormalTok{(kableExtra)}
\end{Highlighting}
\end{Shaded}

\begin{verbatim}
## 
## Attachement du package : 'kableExtra'
\end{verbatim}

\begin{verbatim}
## Les objets suivants sont masqués depuis 'package:flextable':
## 
##     as_image, footnote
\end{verbatim}

\begin{verbatim}
## L'objet suivant est masqué depuis 'package:dplyr':
## 
##     group_rows
\end{verbatim}

\begin{Shaded}
\begin{Highlighting}[]
\NormalTok{knitr}\SpecialCharTok{::}\FunctionTok{kable}\NormalTok{(}\FunctionTok{head}\NormalTok{(iris), }\AttributeTok{caption =} \StringTok{"Iris avec kableExtra"}\NormalTok{)}
\end{Highlighting}
\end{Shaded}

\begin{longtable}[]{@{}rrrrl@{}}
\caption{Iris avec kableExtra}\tabularnewline
\toprule\noalign{}
Sepal.Length & Sepal.Width & Petal.Length & Petal.Width & Species \\
\midrule\noalign{}
\endfirsthead
\toprule\noalign{}
Sepal.Length & Sepal.Width & Petal.Length & Petal.Width & Species \\
\midrule\noalign{}
\endhead
\bottomrule\noalign{}
\endlastfoot
5.1 & 3.5 & 1.4 & 0.2 & setosa \\
4.9 & 3.0 & 1.4 & 0.2 & setosa \\
4.7 & 3.2 & 1.3 & 0.2 & setosa \\
4.6 & 3.1 & 1.5 & 0.2 & setosa \\
5.0 & 3.6 & 1.4 & 0.2 & setosa \\
5.4 & 3.9 & 1.7 & 0.4 & setosa \\
\end{longtable}

\begin{Shaded}
\begin{Highlighting}[]
\FunctionTok{library}\NormalTok{(DT)}
\FunctionTok{datatable}\NormalTok{(iris)}
\CommentTok{\# Cela permet de filtrer, trier, rechercher dans le tableau (pour HTML uniquement)}
\end{Highlighting}
\end{Shaded}

\section{Les graphiques}\label{les-graphiques}

Il faut d'abord installé les packages ggplot2, cowplot, patchwork,
plotly,etc.

\begin{Shaded}
\begin{Highlighting}[]
\FunctionTok{plot}\NormalTok{(mtcars}\SpecialCharTok{$}\NormalTok{mpg, mtcars}\SpecialCharTok{$}\NormalTok{hp, }
     \AttributeTok{main =} \StringTok{"Puissance vs Consommation"}\NormalTok{,}
     \AttributeTok{xlab =} \StringTok{"mpg"}\NormalTok{,}\AttributeTok{ylab=}\StringTok{"hp"}\NormalTok{)}
\end{Highlighting}
\end{Shaded}

\pandocbounded{\includegraphics[keepaspectratio]{Introduction_Rmarkdown_files/figure-latex/unnamed-chunk-14-1.pdf}}

Cela produit un nuage de points

\begin{Shaded}
\begin{Highlighting}[]
\FunctionTok{hist}\NormalTok{(mtcars}\SpecialCharTok{$}\NormalTok{mpg, }\AttributeTok{col =} \StringTok{"skyblue"}\NormalTok{, }
     \AttributeTok{main =} \StringTok{"Histogramme des mpg"}\NormalTok{,}
     \AttributeTok{xlab =} \StringTok{"Miles per gallon"}\NormalTok{)}
\end{Highlighting}
\end{Shaded}

\pandocbounded{\includegraphics[keepaspectratio]{Introduction_Rmarkdown_files/figure-latex/unnamed-chunk-15-1.pdf}}

Ceci est un histogramme

\begin{Shaded}
\begin{Highlighting}[]
\NormalTok{ggplot2}\SpecialCharTok{::}\FunctionTok{ggplot}\NormalTok{(mtcars, }\FunctionTok{aes}\NormalTok{(}\AttributeTok{x =}\NormalTok{ mpg, }\AttributeTok{y =}\NormalTok{ hp)) }\SpecialCharTok{+}
  \FunctionTok{geom\_point}\NormalTok{(}\AttributeTok{color =} \StringTok{"tomato"}\NormalTok{) }\SpecialCharTok{+}
  \FunctionTok{labs}\NormalTok{(}\AttributeTok{title =} \StringTok{"Puissance en fonction du mpg"}\NormalTok{,}
       \AttributeTok{x =} \StringTok{"Miles par gallon"}\NormalTok{, }\AttributeTok{y=}\StringTok{"Puissance"}\NormalTok{)}
\end{Highlighting}
\end{Shaded}

\pandocbounded{\includegraphics[keepaspectratio]{Introduction_Rmarkdown_files/figure-latex/unnamed-chunk-16-1.pdf}}

\begin{Shaded}
\begin{Highlighting}[]
\CommentTok{\#Avec le package ggplot2}
\end{Highlighting}
\end{Shaded}

On peut mettre la couleur qu'on veut

\begin{Shaded}
\begin{Highlighting}[]
\CommentTok{\# Affiche les 20 premières couleurs avec leur nom}
\FunctionTok{barplot}\NormalTok{(}\FunctionTok{rep}\NormalTok{(}\DecValTok{1}\NormalTok{, }\DecValTok{20}\NormalTok{), }\AttributeTok{col =} \FunctionTok{colors}\NormalTok{()[}\DecValTok{1}\SpecialCharTok{:}\DecValTok{20}\NormalTok{], }\AttributeTok{names.arg =} \FunctionTok{colors}\NormalTok{()[}\DecValTok{1}\SpecialCharTok{:}\DecValTok{20}\NormalTok{],}\AttributeTok{las=}\DecValTok{2}\NormalTok{)}
\end{Highlighting}
\end{Shaded}

\pandocbounded{\includegraphics[keepaspectratio]{Introduction_Rmarkdown_files/figure-latex/unnamed-chunk-17-1.pdf}}

\begin{Shaded}
\begin{Highlighting}[]
\CommentTok{\# on peut aussi controler la taille et l\textquotesingle{}alignement}

\FunctionTok{ggplot}\NormalTok{(mtcars, }\FunctionTok{aes}\NormalTok{(}\AttributeTok{x =}\NormalTok{ mpg)) }\SpecialCharTok{+}
  \FunctionTok{geom\_histogram}\NormalTok{(}\AttributeTok{binwidth =} \DecValTok{2}\NormalTok{, }\AttributeTok{fill =} \StringTok{"steelblue"}\NormalTok{) }\SpecialCharTok{+}
  \FunctionTok{labs}\NormalTok{(}\AttributeTok{title =} \StringTok{"Distribution des mpg"}\NormalTok{)}
\end{Highlighting}
\end{Shaded}

\begin{center}\includegraphics{Introduction_Rmarkdown_files/figure-latex/unnamed-chunk-18-1} \end{center}

\begin{Shaded}
\begin{Highlighting}[]
\CommentTok{\# Nous pouvons aussi avoir plusieurs graphiques cote à cote}
\FunctionTok{par}\NormalTok{(}\AttributeTok{mfrow =} \FunctionTok{c}\NormalTok{(}\DecValTok{1}\NormalTok{, }\DecValTok{2}\NormalTok{))  }\CommentTok{\# 1 ligne, 2 colonnes}
\FunctionTok{hist}\NormalTok{(mtcars}\SpecialCharTok{$}\NormalTok{mpg)}
\FunctionTok{boxplot}\NormalTok{(mtcars}\SpecialCharTok{$}\NormalTok{hp)}
\end{Highlighting}
\end{Shaded}

\pandocbounded{\includegraphics[keepaspectratio]{Introduction_Rmarkdown_files/figure-latex/unnamed-chunk-19-1.pdf}}

Pour avoir plus d'informations, vous pouvez partir dans
{[}\url{https://r-graph-gallery.com/}{]}

\subsection{Citation et bibliographie}\label{citation-et-bibliographie}

\subsubsection{Introduction}\label{introduction}

Selon @smith2020, la statistique est essentielle dans l'analyse des
données.

Voir aussi l'étude de @doe2019 pour un autre point de vue.

\subsubsection{Conclusion}\label{conclusion}

La bibliographie s'affiche ci-dessous automatiquement.

\subsubsection{Références}\label{ruxe9fuxe9rences}

\end{document}
